%%%%%%%%%%%%%%%%%%%%%%%%%%%%%%%%%%%%%%%%%
% Wenneker Assignment
% LaTeX Template
% Version 2.0 (12/1/2019)
%
% This template originates from:
% http://www.LaTeXTemplates.com
%
% Authors:
% Vel (vel@LaTeXTemplates.com)
% Frits Wenneker
%
% License:
% CC BY-NC-SA 3.0 (http://creativecommons.org/licenses/by-nc-sa/3.0/)
% 
%%%%%%%%%%%%%%%%%%%%%%%%%%%%%%%%%%%%%%%%%

%----------------------------------------------------------------------------------------
%	PACKAGES AND OTHER DOCUMENT CONFIGURATIONS
%----------------------------------------------------------------------------------------

\documentclass[11pt]{scrartcl} % Font size

\input{structure.tex} % Include the file specifying the document structure and custom commands

%----------------------------------------------------------------------------------------
%	TITLE SECTION
%----------------------------------------------------------------------------------------

\title{	
	\normalfont\normalsize
	\textsc{\Huge Physics-1 Lab SM203P}\\ % Your university, school and/or department name(s)
	\vspace{25pt} % Whitespace
	\rule{\linewidth}{0.5pt}\\ % Thin top horizontal rule
	\vspace{20pt} % Whitespace
	{\huge Experiment 2: Experiments with simple and double pendulums with phase portraits}\\ % The assignment title
	\vspace{12pt} % Whitespace
	\rule{\linewidth}{2pt}\\ % Thick bottom horizontal rule
	\vspace{12pt} % Whitespace
}

\author{\Huge Group-16\\
\\
\LARGE Mayank Chadha(IMT2020045)\\
\\
\LARGE Anshul Jindal(IMT2020039)\\
\\
\LARGE Rahul Jain(IMT2020117)\\
\\
\LARGE Karanjit Saha(IMT2020003)\\
\\
\LARGE Chinmay Parekh(IMT2020069)\\
\\
\LARGE Shashank Shekhar(IMT2020112)} % Your name

\date{\normalsize\today} % Today's date (\today) or a custom date

\begin{document}

\maketitle % Print the title
%----------------------------------------------------------------------------------------
%	FIGURE EXAMPLE
%----------------------------------------------------------------------------------------

\begin{figure}[h] % [h] forces the figure to be output where it is defined in the code (it suppresses floating)
	\centering
	\includegraphics[width=\textwidth, height=5cm]{first.jpg} % Example image
	\caption{Physics lab.}
\end{figure}

%------------------------------------------------
\section{Aim of Experiment}
1. Find g using your phone as a simple pendulum. \par
2. Showing phase portraits and plotting time series for single and double pendulums.\par
3. Calculate percentage dissipation of energy is presence of drag. \par

\section{Apparatus required for the Experiment}
\begin{enumerate}
	\item A string
	\item Phyphox app
	\item Mobile phone or any solid object
	\item Simulation enviornment (https://www.myphysicslab.com/)
\end{enumerate}





%----------------------------------------------------------------------------------------
%	TEXT EXAMPLE
%----------------------------------------------------------------------------------------

%------------------------------------------------
\section{Theory}


\subsection{Acceleration Due to Gravity for Single Pendulum(g)}

We will use Phyphox app and our mobile as a solid object tied with a string and hung like a pendulum and g shown shall be reported. Instead of mobile, we can also use any solid object (such as a stone etc.). In this case, we will note down the time period of our oscillation and the length of the string. Thereafter, to calculate g we will use a well known relation between time period of simple pendulum and g: \par

\begin{align} 
	\begin{split}
		T = 2\pi\sqrt{\frac{L}{g}}\\
	\end{split}					
\end{align}

Rearranging the terms, we get:
\begin{align} 
	\begin{split}
		g = \frac{4\pi^2L}{T^2}\\
	\end{split}					
\end{align}

where\\ g = acceleration due to gravity.\\ T= Time period of simple pendulum.\\ L= Length of string. \par

%------------------------------------------------

\subsection{Phase Portraits}
Phase portrait is the graph between..... $\theta$ For simple pendulum with small amplitude(i.e. $\sin\theta \approx \theta$), phase portrait is generally an ellipse in an ideal case. In the presence of viscous forces such as drag etc. then we observe phase portrait as concentric ellipses. \par
In this case we can quantify energy lost per cycle in terms of decrease in area of 2 consecutive concentric ellipses (percentage decrease).

Formula for the \textbf{area(A)} of the ellipse with major axis as \textbf{a} and minor axis as \textbf{b} is:
\begin{align} 
	\begin{split}
		A = \pi ab\\
	\end{split}					
\end{align}

\subsubsection{Liouville's Theorem}
For Non-Dissipative systems, average phase space traversed is conserved.

%------------------------------------------------
\subsection{Time Series}
Time Series is the graph between .... For non-dissipative systems, amplitude remains constant throughout and no energy is lost whereas for dissipative systems, amplitude keeps on decreasing with time and energy also decreases along.

\subsection{Double Pendulum}
Lagrangian equation for double pendulum is as follows:

%-----------------MAYANK CHADHA WRITE ALL EQUATIONS HERE_______________________%




\section{Sources of Error}
\begin{enumerate}
	\item Air Resistance
	\item Old Tennis ball
	\item Rigged Meter Scale
	\item Inclined or Rough Surface
	\item The value of e may change with the height.
	\item Time taken during collision will also contribute to the wrong calculation of e
	\item Phyphox mobile app uses various sensors which are present in a mobile phone, which may not give accurate results every time.
	\item Electric and magnetic fields may disturb the working of the sensors due to which the results observed may no be completely accurate.
	\item Acoustic stopwatch of the Phyphox app also takes some time to stop the timer after hearing the sound of the collision.
	\item Random error.
\end{enumerate}
%----------------------------------------------------------------------------------------
%	EQUATION EXAMPLES
%----------------------------------------------------------------------------------------
\newpage

\section{Observation and Calculation}
\subsection{Mayank Chadha(IMT2020045)}

\begin{table}[h] % [h] forces the table to be output where it is defined in the code (it suppresses floating)
	\centering % Centre the table
	\begin{tabular}{l l l}
		\toprule
		\textbf{H} & \textbf{h} & \textbf{e} \\
		\midrule
		2.00 m & 1.07 m & 0.53\\
        2.00 m & 1.03 m & 0.51\\
        2.00 m & 1.05 m & 0.52\\
        2.00 m & 1.05 m & 0.52\\
        2.00 m & 1.07 m & 0.53\\
		\bottomrule
	\end{tabular}
	\caption{Observation Table for e}
\end{table}
So the final $h_{avg}$ = 1.05 m\\
So the final $e_{avg}$ = 0.52\\
So the error in e calculated using the equation above is \partial{e} = 0.008 

\begin{table}[h]
\centering
\begin{tabular}{||c c c c c c||} 
\toprule
 \hline
 H & t_0 & t_1 & t_2 & t_n^* & g \\ [0.5ex] 
 \midrule
 \hline\hline
 2.00 m & 0.924 s & 0.679 s  & 0.436 s & 2.039 s & 10.58 $m/s^2$  \\ 
 \hline
 2.00 m & 0.891 s & 0.665 s & 0.448 s & 1.994 s & 10.33 $m/s^2$  \\
 \hline
 2.00 m & 0.924 s & 0.669 s & 0.436 s & 2.029 s  & 10.32 $m/s^2$   \\
 \hline
 2.00 m & 0.961 s & 0.682 s & 0.411 s & 2.054 s  & 10.08 $m/s^2$   \\
 \hline
 2.00 m & 0.940 s & 0.696 s & 0.412 s & 2.048 s  & 10.48 $m/s^2$  \\ [1ex]
 \bottomrule
 \hline
\end{tabular}
\caption{Observation Table for g}
\end{table}
Phyphox absolute value of g = 9.98 $m/s^2$ \\
So the final $t_n^*_{avg}$ = 2.0328 s\\
So the final $g_{avg}$ = 10.36 $m/s^2$\\
So the error in g calculated using the equation above is \partial{g} = 0.872  $m/s^2$\\

\newpage
\subsection{Anshul Jindal(IMT2020039)}

\begin{table}[h] % [h] forces the table to be output where it is defined in the code (it suppresses floating)
	\centering % Centre the table
	\begin{tabular}{l l l}
		\toprule
		\textbf{H} & \textbf{h} & \textbf{e} \\
		\midrule
		1.60 m & 0.84 m & 0.52\\
        1.60 m & 0.86 m & 0.54\\
        1.60 m & 0.84 m & 0.53\\
        1.60 m & 0.82 m & 0.51 \\
        1.60 m & 0.84 m & 0.53 \\
		\bottomrule
	\end{tabular}
	\caption{Observation Table for e}
\end{table}
So the final $h_{avg}$ = 0.84 m\\
So the final $e_{avg}$ = 0.53\\
So the error in e calculated using the equation above is \partial{e} = 0.0096 

\begin{table}[h]
\centering
\begin{tabular}{||c c c c c c||} 
\toprule
 \hline
 H & t_0 & t_1 & t_2 & t_n^* & g \\ [0.5ex] 
 \midrule
 \hline\hline
 1.6 m & 0.786 s & 0.597 s  & 0.435 s & 1.818 s & 10.29 $m/s^2$ \\
 \hline
 1.6 m & 0.802 s & 0.579 s & 0.466 s & 1.847 s & 10.40 $m/s^2$  \\
 \hline
 1.6 m & 0.812 s & 0.606 s & 0.413 s & 1.831 s  & 10.14 $m/s^2$ \\
 \hline
 1.6 m & 0.831 s & 0.585 s & 0.392 s & 1.808 s  & 9.96 $m/s^2$  \\
 \hline
 1.6 m & 0.791 s & 0.572 s & 0.461 s & 1.824 s  & 10.22 $m/s^2$ \\ [1ex]
 \bottomrule
 \hline
\end{tabular}
\caption{Observation Table for g}
\end{table}
Phyphox absolute value of g = 9.82 $m/s^2$\\
So the final $t_n^*_{avg}$ = 1.8256 s\\
So the final $g_{avg}$ = 10.20 $m/s^2$\\
So the error in g calculated using the equation above is \partial{g} = 1.105  $m/s^2$\\

\newpage
\subsection{Rahul Jain(IMT2020117)}

\begin{table}[h] % [h] forces the table to be output where it is defined in the code (it suppresses floating)
	\centering % Centre the table
	\begin{tabular}{l l l}
		\toprule
		\textbf{H} & \textbf{h} & \textbf{e} \\
		\midrule
		1.75 m & 0.83 m & 0.47\\
        1.75 m & 0.87 m  & 0.48\\
        1.75 m & 0.83 m  & 0.47\\
        1.75 m & 0.86 m & 0.49 \\
        1.75 m & 0.84 m & 0.48 \\
		\bottomrule
	\end{tabular}
	\caption{Observation Table for e}
\end{table}
So the final $h_{avg}$ = 0.85 m\\
So the final $e_{avg}$ = 0.48\\
So the error in e calculated using the equation above is \partial{e} = 0.008

\begin{table}[h]
\centering
\begin{tabular}{||c c c c c c||} 
\toprule
 \hline
 H & t_0 & t_1 & t_2 & t_n^* & g \\ [0.5ex] 
 \midrule
 \hline\hline
1.75 m & 0.651 s & 0.576 s  & 0.503 s & 1.730 s & 10.56 $m/s^2$  \\ 
 \hline
 1.75 m & 0.710 s & 0.592 s & 0.464 s & 1.766 s & 10.603 $m/s^2$  \\
 \hline
 1.75 m & 0.655 s & 0.540 s & 0.487 s & 1.682 s  & 10.98 $m/s^2$   \\
 \hline
 1.75 m & 0.635 s & 0.605 s & 0.468 s & 1.708 s  & 10.88 $m/s^2$   \\
 \hline
 1.75 m & 0.638 s & 0.565 s & 0.480 s & 1.683 s  & 10.97 $m/s^2$  \\ [1ex]
 \bottomrule
 \hline
\end{tabular}
\caption{Observation Table for g}
\end{table}
Phyphox absolute value of g = 9.76 $m/s^2$\\
So the final $t_n^*_{avg}$ = 1.71 s\\
So the final $g_{avg}$ = 10.80 $m/s^2$\\
So the error in g calculated using the equation above is \partial{g} = 0.972 $m/s^2$ \\

\newpage
\subsection{Karanjit Saha(IMT2020003)}

\begin{table}[h] % [h] forces the table to be output where it is defined in the code (it suppresses floating)
	\centering % Centre the table
	\begin{tabular}{l l l}
		\toprule
		\textbf{H} & \textbf{h} & \textbf{e} \\
		\midrule
		2.00 m & 1.02 m & 0.51\\
		2.00 m & 0.99 m  & 0.50\\
		2.00 m & 0.98 m  & 0.49\\
		2.00 m & 1.01 m & 0.50 \\
		2.00 m & 1.00 m & 0.50 \\
		\bottomrule
	\end{tabular}
	\caption{Observation Table for e}
\end{table}
So the final $h_{avg}$ = 1.00 m\\
So the final $e_{avg}$ = 0.50\\
So the error in e calculated using the equation above is \partial{e} = 0.001 

\begin{table}[h]
\centering
\begin{tabular}{||c c c c c c||} 
\toprule
 \hline
 H & t_0 & t_1 & t_2 & t_n^* & g \\ [0.5ex] 
 \midrule
 \hline\hline
 2.00 m & 0.917 s & 0.685 s  & 0.527 s & 2.129 s & 8.60 $m/s^2$  \\ 
 \hline
 2.00 m & 0.889 s & 0.710 s & 0.535 s & 2.134 s & 8.56 $m/s^2$  \\
 \hline
 2.00 m & 0.916 s & 0.681 s & 0.530 s & 2.127 s  & 8.61 $m/s^2$   \\
 \hline
 2.00 m & 0.912 s & 0.700 s & 0.519 s & 2.131 s  & 8.58 $m/s^2$   \\
 \hline
 2.00 m & 0.910 s & 0.690 s & 0.531 s & 2.131 s  & 8.58 $m/s^2$  \\ [1ex] 
 \bottomrule
 \hline
\end{tabular}
\caption{Observation Table for g}
\end{table}
Phyphox absolute value of g = 9.27 $m/s^2$\\
So the final $t_n^*_{avg}$ = 2.130 s\\
So the final $g_{avg}$ = 8.59 $m/s^2$\\
So the error in g calculated using the equation above is \partial{g} = 0.702 $m/s^2$

\newpage
\subsection{Chinmay Parekh(IMT2020069)}

\begin{table}[h] % [h] forces the table to be output where it is defined in the code (it suppresses floating)
	\centering % Centre the table
	\begin{tabular}{l l l}
		\toprule
		\textbf{H} & \textbf{h} & \textbf{e} \\
		\midrule
		2.00 m & 1.10 m & 0.55\\
        2.00 m & 1.09 m  & 0.54\\
        2.00 m & 1.08 m  & 0.54\\
        2.00 m & 1.10 m & 0.55 \\
        2.00 m & 1.06 m & 0.53 \\
		\bottomrule
	\end{tabular}
	\caption{Observation Table for e}
\end{table}
So the final $h_{avg}$ = 1.09 m\\
So the final $e_{avg}$ = 0.54\\
So the error in e calculated using the equation above is \partial{e} = 0.007

\begin{table}[h]
\centering
\begin{tabular}{||c c c c c c||} 
\toprule
 \hline
 H & t_0 & t_1 & t_2 & t_n^* & g \\ [0.5ex] 
 \midrule
 \hline\hline
 2.00 m & 0.751 s & 0.698 s  & 0.648 s & 2.097 s & 10.51 $m/s^2$  \\ 
 \hline
 2.00 m & 0.787 s & 0.605 s & 0.598 s & 1.989 s & 11.48 $m/s^2$  \\
 \hline
 2.00 m & 0.753 s & 0.690 s & 0.604 s & 2.047 s  & 10.67 $m/s^2$   \\
 \hline
 2.00 m & 0.762 s & 0.685 s & 0.623 s & 2.07 s  & 10.78 $m/s^2$   \\
 \hline
 2.00 m & 0.771 s & 0.690 s & 0.611 s & 2.072 s  & 10.07 $m/s^2$  \\ [1ex]
 \bottomrule
 \hline
\end{tabular}
\caption{Observation Table for g}
\end{table}
Phyphox absolute value of g = 10.03 $m/s^2$\\
So the final $t_n^*_{avg}$ = 2.055 s\\
So the final $g_{avg}$ = 10.70 $m/s^2$\\
So the error in g calculated using the equation above is \partial{g} = 0.935 $m/s^2$\\

\newpage
\subsection{Shashank Shekhar(IMT2020112)}

\begin{table}[h] % [h] forces the table to be output where it is defined in the code (it suppresses floating)
	\centering % Centre the table
	\begin{tabular}{l l l}
		\toprule
		\textbf{H} & \textbf{h} & \textbf{e} \\
		\midrule
		2.00 m & 1.01 m & 0.50\\
        2.00 m & 0.96 m  & 0.48\\
        2.00 m & 0.98 m  & 0.49\\
        2.00 m & 1.05 m & 0.52 \\
        2.00 m & 0.99 m & 0.50 \\
		\bottomrule
	\end{tabular}
	\caption{Observation Table for e}
\end{table}
So the final $h_{avg}$ = 0.99 m\\
So the final $e_{avg}$ = 0.49\\
So the error in e calculated using the equation above is \partial{e} = 0.001

\begin{table}[h]
\centering
\begin{tabular}{||c c c c c c||} 
\toprule
 \hline
 H & t_0 & t_1 & t_2 & t_n^* & g \\ [0.5ex] 
 \midrule
 \hline\hline
 2.00 m & 0.917 s & 0.687 s  & 0.525 s & 2.129 s & 8.57 $m/s^2$  \\ 
 \hline
 2.00 m & 0.889 s & 0.667 s & 0.505 s & 2.061 s & 9.14 $m/s^2$  \\
 \hline 
 2.00 m & 0.914 s & 0.656 s & 0.502 s & 2.072 s  & 9.04 $m/s^2$   \\
 \hline
 2.00 m & 0.912 s & 0.702 s & 0.518 s & 2.132 s  & 8.54 $m/s^2$   \\
 \hline
 2.00 m & 0.917 s & 0.688 s & 0.509 s & 2.114 s  & 8.68 $m/s^2$  \\ 
 [1ex]
 \bottomrule
 \hline
\end{tabular}
\caption{Observation Table for g}
\end{table}
Phyphox absolute value of g = 9.53 $m/s^2$\\
So the final $t_n^*_{avg}$ = 2.102 s\\
So the final $g_{avg}$ = 8.79 $m/s^2$\\
So the error in g calculated using the equation above is \partial{g} = 0.009 $m/s^2$\\

\newpage

\end{document}
