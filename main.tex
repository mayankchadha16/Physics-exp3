%%%%%%%%%%%%%%%%%%%%%%%%%%%%%%%%%%%%%%%%%
% Wenneker Assignment
% LaTeX Template
% Version 2.0 (12/1/2019)
%
% This template originates from:
% http://www.LaTeXTemplates.com
%
% Authors:
% Vel (vel@LaTeXTemplates.com)
% Frits Wenneker
%
% License:
% CC BY-NC-SA 3.0 (http://creativecommons.org/licenses/by-nc-sa/3.0/)
% 
%%%%%%%%%%%%%%%%%%%%%%%%%%%%%%%%%%%%%%%%%

%----------------------------------------------------------------------------------------
%	PACKAGES AND OTHER DOCUMENT CONFIGURATIONS
%----------------------------------------------------------------------------------------

\documentclass[11pt]{scrartcl} % Font size

\input{structure.tex} % Include the file specifying the document structure and custom commands

%----------------------------------------------------------------------------------------
%	TITLE SECTION
%----------------------------------------------------------------------------------------

\title{	
	\normalfont\normalsize
	\textsc{\Huge Physics-1 Lab SM203P}\\ % Your university, school and/or department name(s)
	\vspace{25pt} % Whitespace
	\rule{\linewidth}{0.5pt}\\ % Thin top horizontal rule
	\vspace{20pt} % Whitespace
	{\huge Experiment 2: Experiments with simple and double pendulums with phase portraits}\\ % The assignment title
	\vspace{12pt} % Whitespace
	\rule{\linewidth}{2pt}\\ % Thick bottom horizontal rule
	\vspace{12pt} % Whitespace
}

\author{\Huge Group-16\\
\\
\LARGE Mayank Chadha(IMT2020045)\\
\\
\LARGE Anshul Jindal(IMT2020039)\\
\\
\LARGE Rahul Jain(IMT2020117)\\
\\
\LARGE Karanjit Saha(IMT2020003)\\
\\
\LARGE Chinmay Parekh(IMT2020069)\\
\\
\LARGE Shashank Shekhar(IMT2020112)} % Your name

\date{\normalsize\today} % Today's date (\today) or a custom date

\begin{document}

\maketitle % Print the title
%----------------------------------------------------------------------------------------
%	FIGURE EXAMPLE
%----------------------------------------------------------------------------------------

\begin{figure}[h] % [h] forces the figure to be output where it is defined in the code (it suppresses floating)
	\centering
	\includegraphics[width=\textwidth, height=5cm]{first.jpg} % Example image
	\caption{Physics lab.}
\end{figure}

%------------------------------------------------
\section{Aim of Experiment}
1. To study the equation $x_{n+1} =$ r$x_n(1−x_n)$ and to see how it varies with r.\par
2.  To plot “orbit diagram” or “bifurcation diagram”. \par
3. To calculate Feigenbaum constant.  \par




%----------------------------------------------------------------------------------------
%	TEXT EXAMPLE
%----------------------------------------------------------------------------------------

%------------------------------------------------
\section{Theory}


\subsection{Acceleration Due to Gravity for Single Pendulum(g)}

We will use the Phyphox app and our mobile as a solid object tied with a string and hung like a pendulum, and the g shown shall be reported. Instead of mobile, we can also use any solid object (such as a stone ). In this case, we will note down the period of our oscillation and the length of the string. After that, to calculate g, we will use a well-known relation between the period of simple pendulum and g: \par

\begin{align} 
	\begin{split}
		T = 2\pi\sqrt{\frac{L}{g}}\\
	\end{split}					
\end{align}

Rearranging the terms, we get:
\begin{align} 
	\begin{split}
		g = \frac{4\pi^2L}{T^2}\\
	\end{split}					
\end{align}

where\\ g = acceleration due to gravity.\\ T= Time period of simple pendulum.\\ L= Length of string. \par

%------------------------------------------------

\subsection{Phase Portraits}
Phase portrait is the graph between..... $\theta$ For a simple pendulum with small amplitude(i.e. $\sin\theta \approx \theta$), phase portrait is generally an \textbf{ellipse} in an ideal case. As we increase the amplitude of the pendulum ($\theta$), we will observe the increase in length of both major and minor axis of the ellipse. \\
In the presence of viscous forces such as drag, then we observe phase portrait as concentric ellipses. In this case, we can quantify energy lost per cycle in the decrease in the area of 2 consecutive concentric ellipses (percentage decrease).

Formula for the \textbf{area(A)} of the ellipse with major axis as \textbf{a} and minor axis as \textbf{b} is:
\begin{align} 
	\begin{split}
		A = \pi ab \nonumber\\
	\end{split}					
\end{align}

\subsubsection{Liouville's Theorem}
For Non-Dissipative systems, the average phase space traversed is conserved.

%------------------------------------------------
\subsection{Time Series}
Time Series is the graph between... For non-dissipative systems, the amplitude remains constant throughout, and no energy is lost, whereas, for dissipative systems, amplitude keeps on decreasing with time, and energy also decreases along.

\subsection{Double Pendulum}
Lagrangian equation for double pendulum is as follows:
\begin{equation}
\left(m_{1}+m_{2}\right) L_{1}^{2} \theta_{1}^{\prime \prime}+m_{2} L_{1} L_{2} \theta_{2}^{\prime \prime} \cos \left(\theta_{1}-\theta_{2}\right)+m_{2} L_{1} L_{2} \theta_{1}^{\prime} \theta_{2}^{\prime} \sin \left(\theta_{1}-\theta_{2}\right)+g L_{1} \sin \theta_{1}\left(m_{1}+m_{2}\right)=0
\end{equation}
and
\begin{equation}
m_{2} L_{2}^{2} \theta_{2}^{\prime \prime}+m_{2} L_{1} L_{2} \theta_{1}^{\prime \prime} \cos \left(\theta_{1}-\theta_{2}\right)-m_{2} L_{4} L_{2} \theta_{1}^{\prime 2} \sin \left(\theta_{1}-\theta_{2}\right)+g m_{2} L_{2} \sin \theta_{2}=0
\end{equation}
\\
After Nondimensionalization, i.e substituting in the above equations as follows:
\begin{equation*}
M=\frac{m_{2}}{m_{1}+m_{2}} \hspace{1cm} \mbox{and} \hspace{1cm} L=\frac{l_{2}}{l_{1}}
\end{equation*}

We get:
\begin{equation}
\theta_{1}^{\prime \prime}+M L \theta_{2}^{\prime \prime} \cos \left(\theta_{1}-\theta_{2}\right)+M L \theta_{1}^{\prime} \theta_{2}^{\prime} \sin \left(\theta_{1}-\theta_{2}\right)+\omega^{2} \sin \theta_{1}=0
\end{equation}
and
\begin{equation}
L \theta_{2}^{\prime \prime}+\theta_{1}^{\prime \prime} \cos \left(\theta_{1}-\theta_{2}\right)-\theta_{1}^{\prime 2} \sin \left(\theta_{1}-\theta_{2}\right)+\omega^{2} \sin \theta_{2}=0
\end{equation}
%-----------------MAYANK CHADHA WRITE ALL EQUATIONS HERE_______________________%




\section{Sources of Error}
\begin{enumerate}
	\item Air Resistance
	\item Old Tennis ball
	\item Rigged Meter Scale
	\item Inclined or Rough Surface
	\item The value of e may change with the height.
	\item Time taken during the collision will also contribute to the wrong calculation of e
	\item Phyphox mobile app uses various sensors present in a mobile phone, which may not give accurate results every time.
	\item Electric and magnetic fields may disturb the working of the sensors, due to which the results observed may not be completely accurate.
	\item Acoustic stopwatch of the Phyphox app also takes some time to stop the timer after hearing the sound of the collision.
	\item Random error.
\end{enumerate}
%----------------------------------------------------------------------------------------
%	EQUATION EXAMPLES
%----------------------------------------------------------------------------------------
\newpage

\section{Observation and Calculation}
\textbf{Q1} Common Code for the group:
\begin{lstlisting}[language=Python, caption= Code for plotting X_n vs n graph for a given r]
import matplotlib.pyplot as plt
import numpy as np

# l=[]
# l.append(0.1)
# r = int(input())
# for x in range (0,400,1): 
#     l.append(r*l[x]*(1-l[x]))
#     plt.plot(x,l[x])
# plt.show()

n = []
x_limit=150
for i in range (x_limit):
    n.append(i)

y = []
y.append(0.1)
k = 0.1

r = float(input())

for i in range (x_limit-1):
    k = r * k * (1 - k)
    y.append(k)


plt.xlabel("n")
plt.ylabel("$X_n$")
plt.title("r="+str(r))
plt.plot(n, y)#,c='g')
plt.show() 
\end{lstlisting}
\newpage
\subsection{Mayank Chadha(IMT2020045)}
\textbf{Q2}.
\begin{figure}[h] % [h] forces the figure to be output where it is defined in the code (it suppresses floating)
	\centering
	\includegraphics[width=12cm, height=8cm]{Mayank.png} % Example image
	\caption {Plot for $x_n$ vs n graph for r= 3.853045}
\end{figure}
\subsection{Anshul Jindal(IMT2020039)}
\textbf{Q2}.
\begin{figure}[h] % [h] forces the figure to be output where it is defined in the code (it suppresses floating)
	\centering
	\includegraphics[width=12cm, height=8cm]{anshul_bsdk.png} % Example image
	\caption {Plot for $x_n$ vs n graph for r= 3.853045}
\end{figure}
\newpage
\subsection{Rahul Jain (IMT2020117)}
\textbf{Q2}.
\begin{figure}[h] % [h] forces the figure to be output where it is defined in the code (it suppresses floating)
	\centering
	\includegraphics[width=12cm, height=8cm]{Rahul_singer.png} % Example image
	\caption {Plot for $x_n$ vs n graph for r= 3.853117}
\end{figure}
\newpage
\subsection{Karanjit Saha (IMT2020003)}
\textbf{Q2}.
\begin{figure}[h] % [h] forces the figure to be output where it is defined in the code (it suppresses floating)
	\centering
	\includegraphics[width=12cm, height=8cm]{Karanjit.png} % Example image
	\caption {Plot for $x_n$ vs n graph for r= 3.853003}
\end{figure}
\newpage
\subsection{Chinmay Parekh (IMT2020069)}
\textbf{Q2}.
\begin{figure}[h] % [h] forces the figure to be output where it is defined in the code (it suppresses floating)
	\centering
	\includegraphics[width=12cm, height=8cm]{chinmay69.png} % Example image
	\caption {Plot for $x_n$ vs n graph for r= 3.853069}
\end{figure}
\newpage
\subsection{Shashank Shekhar (IMT2020112)}
\textbf{Q2}.
\begin{figure}[h] % [h] forces the figure to be output where it is defined in the code (it suppresses floating)
	\centering
	\includegraphics[width=12cm, height=8cm]{shashank.png} % Example image
	\caption {Plot for $x_n$ vs n graph for r= 3.853112}
\end{figure}
\end{document}