%%%%%%%%%%%%%%%%%%%%%%%%%%%%%%%%%%%%%%%%%
% Wenneker Assignment
% LaTeX Template
% Version 2.0 (12/1/2019)
%
% This template originates from:
% http://www.LaTeXTemplates.com
%
% Authors:
% Vel (vel@LaTeXTemplates.com)
% Frits Wenneker
%
% License:
% CC BY-NC-SA 3.0 (http://creativecommons.org/licenses/by-nc-sa/3.0/)
% 
%%%%%%%%%%%%%%%%%%%%%%%%%%%%%%%%%%%%%%%%%

%----------------------------------------------------------------------------------------
%	PACKAGES AND OTHER DOCUMENT CONFIGURATIONS
%----------------------------------------------------------------------------------------

\documentclass[11pt]{scrartcl} % Font size

\input{structure.tex} % Include the file specifying the document structure and custom commands

%----------------------------------------------------------------------------------------
%	TITLE SECTION
%----------------------------------------------------------------------------------------

\title{	
	\normalfont\normalsize
	\textsc{\Huge Physics-1 Lab SM203P}\\ % Your university, school and/or department name(s)
	\vspace{25pt} % Whitespace
	\rule{\linewidth}{0.5pt}\\ % Thin top horizontal rule
	\vspace{20pt} % Whitespace
	{\huge Experiment 2:Find g using your phone as a simple pendulum – using the “pendulum” option in the PhyPhox app.}\\ % The assignment title
	\vspace{12pt} % Whitespace
	\rule{\linewidth}{2pt}\\ % Thick bottom horizontal rule
	\vspace{12pt} % Whitespace
}

\author{\Huge Group-16\\
\\
\LARGE Mayank Chadha(IMT2020045)\\
\\
\LARGE Anshul Jindal(IMT2020039)\\
\\
\LARGE Rahul Jain(IMT2020117)\\
\\
\LARGE Karanjit Saha(IMT2020003)\\
\\
\LARGE Chinmay Parekh(IMT2020069)\\
\\
\LARGE Shashank Shekhar(IMT2020112)} % Your name

\date{\normalsize\today} % Today's date (\today) or a custom date

\begin{document}

\maketitle % Print the title
%----------------------------------------------------------------------------------------
%	FIGURE EXAMPLE
%----------------------------------------------------------------------------------------

\begin{figure}[h] % [h] forces the figure to be output where it is defined in the code (it suppresses floating)
	\centering
	\includegraphics[width=\textwidth, height=5cm]{first.jpg} % Example image
	\caption{Physics lab.}
\end{figure}

%------------------------------------------------
\section{Aim of Experiment}
Find g using your phone as a simple pendulum – using the “pendulum” option in the PhyPhox app. \par

\section{Apparatus required for the Experiment}
\begin{enumerate}
	\item A string
	\item Phyphox mobile app
\end{enumerate}





%----------------------------------------------------------------------------------------
%	TEXT EXAMPLE
%----------------------------------------------------------------------------------------

%------------------------------------------------
\section{Theory}
Every experiment uses some or the other instrument to make several measurements, and none of them is perfect. Every instrument has some error associated with it. These errors can be broadly classified into two types, namely systematic errors and random errors.\par 
\textbf{Systematic errors}: - The errors are caused due to imperfect measurement techniques or imperfect apparatus. \\
\textbf{Random errors}: - The errors in measurement are caused by factors that from one measurement to another. \par 

Since obliterating these errors is impossible in real-life scenarios. Hence we need to measure how accurate our obtained result is, precisely what we do in error analysis. For doing error analysis, we first need to comprehend the two critical definitions, which are as mentioned below: - \par 

\textbf{A) Accuracy}: - The closeness of the result (or the practical values) to the theoretically calculated values is represented by this quantity. The more accurate our experiment is, the closer is the practical value to the theoretical value.\par

\textbf{B) Precision}: - Precision represents the closeness of the different results obtained by repeating the same experiments in the same situation along with the same apparatus. \par

\subsection{Acceleration Due to Gravity(g)}

As we know from Newton’s motion of equations, that if an object is dropped (i.e., initial velocity =0) from some height ‘h’ (such that h is negligibly small compared to the radius of the earth), then  \par

\begin{align} 
	\begin{split}
		h &= \frac{1}{2}gt^2\\
	\end{split}					
\end{align}

where g = acceleration due to gravity and t= time taken for the object to reach the ground. \par

\subsubsection{Procedure (for calculating g)}
For finding g, we follow the below-mentioned steps: - \par

A) Select a random number ‘n’ such that the ball will be able to bounce at least n number of times. Also, select a height H (which will be constant throughout the experiment)\par

B) Drop the ball from H height, record the total time it takes to do n bounces, and name it $T_n^*$. \par

C) Fill the below mentioned observation table and use the formula mentioned below to use the previously fetched e and the respective H and $T_n^*$ to calculate g. \par
\begin{align} 
	\begin{split}
		g &= \frac{8He}{t_{n}^{*2}}\left(\dfrac{(\sqrt{e})^n-1}{\sqrt{e}-1}\right)^2\\
	\end{split}					
\end{align}
	 
D) The final value of g will be the avg of all the g’s obtained in the experiment. \par

\subsubsection{Errors(for g)}
The equation deduced after error analysis on the above equation is:(Assuming n=3)\par
\textbf{Least count for heights h and H = 0.01 m and for time = 0.001 s.}\par

Derivation for the Error Propagation:
\begin{align} 
		g &= \frac{8He}{t_{n}^{*2}}\left(\dfrac{(\sqrt{e})^n-1}{\sqrt{e}-1}\right)^2 \nonumber
\end{align}
    Taking logarithmic on both sides and taking n = 3:
\begin{align} 
            \log_e g &= \log_e H + \log_e t^{-2} + 2(log_e ({e^{3/2}-1}) - log_e ({\sqrt{e}-1})) \nonumber
\end{align}
    Differentiating on both sides we get:
\begin{align} 
        \frac{\partial{g}}{g} &= \frac{\partial{H}}{H} + \frac{\partial{e}}{e} + \frac{2\partial{t_n^*}}{t_n^*} + \left|\frac{3\partial{e^{3/2-1}}}{\sqrt{e}-1} + \frac{\partial{e^{-1/2}}}{\sqrt{e}-1}\right|\partial{e} \nonumber
\end{align}
   On further solving the above equation we get:
\begin{align} 
	\begin{split}
		\frac{\partial{g}}{g} &= \frac{\partial{H}}{H} + \frac{\partial{e}}{e}+\frac{2\partial{t_m^*}}{t_m^*}+\left|\frac{3\sqrt{e}\partial{e}}{e^{3/2}-1}\right|+\left|\frac{\partial{e}}{\sqrt{e}(\sqrt{e}-1)}\right|\\
	\end{split}					
\end{align}
%------------------------------------------------

\subsection{Coefficient of Restitution(e)}
From the definition of the coefficient of restitution we know, 
\begin{align} 
	\begin{split}
		e &= \frac{h}{H}\\
	\end{split}					
\end{align}

%------------------------------------------------
\subsubsection{Procedure (for calculating e)}
For finding e, we follow the below-mentioned steps: - \par

A) Release the ball from some height ‘H’ and note it in the observation table. \par

B) Record the height to which the ball rises after the first collision (let it be ‘h’) and note it in the observation table. \par

C) Calculate e by using the equation mentioned beforehand. \par

D) Repeat the above steps several times to get a good idea about the actual e and fill the observation table accordingly. \par

E) Calculate the average of all the e’s calculated in the above steps. This e is the coefficient of restitution of the ball. \par

\subsubsection{Errors(for e)}
The equation deduced after error analysis on the above equation is:\par
Derivation for the Error Propagation:
\begin{align} 
		e &= \frac{h}{H} \nonumber
\end{align}
Taking log both sides :
\begin{align}
\log_e e &= \log_e h - \log_e H \nonumber
\end{align}
   Finally we get:
\begin{align} 
	\begin{split}
		\frac{\partial{e}}{e} &= \frac{\partial{H}}{H} + \frac{\partial{h}}{h}\\
	\end{split}					
\end{align}

\section{Sources of Error}
\begin{enumerate}
	\item Air Resistance
	\item Old Tennis ball
	\item Rigged Meter Scale
	\item Inclined or Rough Surface
	\item The value of e may change with the height.
	\item Time taken during collision will also contribute to the wrong calculation of e
	\item Phyphox mobile app uses various sensors which are present in a mobile phone, which may not give accurate results every time.
	\item Electric and magnetic fields may disturb the working of the sensors due to which the results observed may no be completely accurate.
	\item Acoustic stopwatch of the Phyphox app also takes some time to stop the timer after hearing the sound of the collision.
	\item Random error.
\end{enumerate}
%----------------------------------------------------------------------------------------
%	EQUATION EXAMPLES
%----------------------------------------------------------------------------------------
\newpage

\section{Observation and Calculation}
\subsection{Mayank Chadha(IMT2020045)}

\begin{table}[h] % [h] forces the table to be output where it is defined in the code (it suppresses floating)
	\centering % Centre the table
	\begin{tabular}{l l l}
		\toprule
		\textbf{H} & \textbf{h} & \textbf{e} \\
		\midrule
		2.00 m & 1.07 m & 0.53\\
        2.00 m & 1.03 m & 0.51\\
        2.00 m & 1.05 m & 0.52\\
        2.00 m & 1.05 m & 0.52\\
        2.00 m & 1.07 m & 0.53\\
		\bottomrule
	\end{tabular}
	\caption{Observation Table for e}
\end{table}
So the final $h_{avg}$ = 1.05 m\\
So the final $e_{avg}$ = 0.52\\
So the error in e calculated using the equation above is \partial{e} = 0.008 

\begin{table}[h]
\centering
\begin{tabular}{||c c c c c c||} 
\toprule
 \hline
 H & t_0 & t_1 & t_2 & t_n^* & g \\ [0.5ex] 
 \midrule
 \hline\hline
 2.00 m & 0.924 s & 0.679 s  & 0.436 s & 2.039 s & 10.58 $m/s^2$  \\ 
 \hline
 2.00 m & 0.891 s & 0.665 s & 0.448 s & 1.994 s & 10.33 $m/s^2$  \\
 \hline
 2.00 m & 0.924 s & 0.669 s & 0.436 s & 2.029 s  & 10.32 $m/s^2$   \\
 \hline
 2.00 m & 0.961 s & 0.682 s & 0.411 s & 2.054 s  & 10.08 $m/s^2$   \\
 \hline
 2.00 m & 0.940 s & 0.696 s & 0.412 s & 2.048 s  & 10.48 $m/s^2$  \\ [1ex]
 \bottomrule
 \hline
\end{tabular}
\caption{Observation Table for g}
\end{table}
Phyphox absolute value of g = 9.98 $m/s^2$ \\
So the final $t_n^*_{avg}$ = 2.0328 s\\
So the final $g_{avg}$ = 10.36 $m/s^2$\\
So the error in g calculated using the equation above is \partial{g} = 0.872  $m/s^2$\\

\newpage
\subsection{Anshul Jindal(IMT2020039)}

\begin{table}[h] % [h] forces the table to be output where it is defined in the code (it suppresses floating)
	\centering % Centre the table
	\begin{tabular}{l l l}
		\toprule
		\textbf{H} & \textbf{h} & \textbf{e} \\
		\midrule
		1.60 m & 0.84 m & 0.52\\
        1.60 m & 0.86 m & 0.54\\
        1.60 m & 0.84 m & 0.53\\
        1.60 m & 0.82 m & 0.51 \\
        1.60 m & 0.84 m & 0.53 \\
		\bottomrule
	\end{tabular}
	\caption{Observation Table for e}
\end{table}
So the final $h_{avg}$ = 0.84 m\\
So the final $e_{avg}$ = 0.53\\
So the error in e calculated using the equation above is \partial{e} = 0.0096 

\begin{table}[h]
\centering
\begin{tabular}{||c c c c c c||} 
\toprule
 \hline
 H & t_0 & t_1 & t_2 & t_n^* & g \\ [0.5ex] 
 \midrule
 \hline\hline
 1.6 m & 0.786 s & 0.597 s  & 0.435 s & 1.818 s & 10.29 $m/s^2$ \\
 \hline
 1.6 m & 0.802 s & 0.579 s & 0.466 s & 1.847 s & 10.40 $m/s^2$  \\
 \hline
 1.6 m & 0.812 s & 0.606 s & 0.413 s & 1.831 s  & 10.14 $m/s^2$ \\
 \hline
 1.6 m & 0.831 s & 0.585 s & 0.392 s & 1.808 s  & 9.96 $m/s^2$  \\
 \hline
 1.6 m & 0.791 s & 0.572 s & 0.461 s & 1.824 s  & 10.22 $m/s^2$ \\ [1ex]
 \bottomrule
 \hline
\end{tabular}
\caption{Observation Table for g}
\end{table}
Phyphox absolute value of g = 9.82 $m/s^2$\\
So the final $t_n^*_{avg}$ = 1.8256 s\\
So the final $g_{avg}$ = 10.20 $m/s^2$\\
So the error in g calculated using the equation above is \partial{g} = 1.105  $m/s^2$\\

\newpage
\subsection{Rahul Jain(IMT2020117)}

\begin{table}[h] % [h] forces the table to be output where it is defined in the code (it suppresses floating)
	\centering % Centre the table
	\begin{tabular}{l l l}
		\toprule
		\textbf{H} & \textbf{h} & \textbf{e} \\
		\midrule
		1.75 m & 0.83 m & 0.47\\
        1.75 m & 0.87 m  & 0.48\\
        1.75 m & 0.83 m  & 0.47\\
        1.75 m & 0.86 m & 0.49 \\
        1.75 m & 0.84 m & 0.48 \\
		\bottomrule
	\end{tabular}
	\caption{Observation Table for e}
\end{table}
So the final $h_{avg}$ = 0.85 m\\
So the final $e_{avg}$ = 0.48\\
So the error in e calculated using the equation above is \partial{e} = 0.008

\begin{table}[h]
\centering
\begin{tabular}{||c c c c c c||} 
\toprule
 \hline
 H & t_0 & t_1 & t_2 & t_n^* & g \\ [0.5ex] 
 \midrule
 \hline\hline
1.75 m & 0.651 s & 0.576 s  & 0.503 s & 1.730 s & 10.56 $m/s^2$  \\ 
 \hline
 1.75 m & 0.710 s & 0.592 s & 0.464 s & 1.766 s & 10.603 $m/s^2$  \\
 \hline
 1.75 m & 0.655 s & 0.540 s & 0.487 s & 1.682 s  & 10.98 $m/s^2$   \\
 \hline
 1.75 m & 0.635 s & 0.605 s & 0.468 s & 1.708 s  & 10.88 $m/s^2$   \\
 \hline
 1.75 m & 0.638 s & 0.565 s & 0.480 s & 1.683 s  & 10.97 $m/s^2$  \\ [1ex]
 \bottomrule
 \hline
\end{tabular}
\caption{Observation Table for g}
\end{table}
Phyphox absolute value of g = 9.76 $m/s^2$\\
So the final $t_n^*_{avg}$ = 1.71 s\\
So the final $g_{avg}$ = 10.80 $m/s^2$\\
So the error in g calculated using the equation above is \partial{g} = 0.972 $m/s^2$ \\

\newpage
\subsection{Karanjit Saha(IMT2020003)}

\begin{table}[h] % [h] forces the table to be output where it is defined in the code (it suppresses floating)
	\centering % Centre the table
	\begin{tabular}{l l l}
		\toprule
		\textbf{H} & \textbf{h} & \textbf{e} \\
		\midrule
		2.00 m & 1.02 m & 0.51\\
		2.00 m & 0.99 m  & 0.50\\
		2.00 m & 0.98 m  & 0.49\\
		2.00 m & 1.01 m & 0.50 \\
		2.00 m & 1.00 m & 0.50 \\
		\bottomrule
	\end{tabular}
	\caption{Observation Table for e}
\end{table}
So the final $h_{avg}$ = 1.00 m\\
So the final $e_{avg}$ = 0.50\\
So the error in e calculated using the equation above is \partial{e} = 0.001 

\begin{table}[h]
\centering
\begin{tabular}{||c c c c c c||} 
\toprule
 \hline
 H & t_0 & t_1 & t_2 & t_n^* & g \\ [0.5ex] 
 \midrule
 \hline\hline
 2.00 m & 0.917 s & 0.685 s  & 0.527 s & 2.129 s & 8.60 $m/s^2$  \\ 
 \hline
 2.00 m & 0.889 s & 0.710 s & 0.535 s & 2.134 s & 8.56 $m/s^2$  \\
 \hline
 2.00 m & 0.916 s & 0.681 s & 0.530 s & 2.127 s  & 8.61 $m/s^2$   \\
 \hline
 2.00 m & 0.912 s & 0.700 s & 0.519 s & 2.131 s  & 8.58 $m/s^2$   \\
 \hline
 2.00 m & 0.910 s & 0.690 s & 0.531 s & 2.131 s  & 8.58 $m/s^2$  \\ [1ex] 
 \bottomrule
 \hline
\end{tabular}
\caption{Observation Table for g}
\end{table}
Phyphox absolute value of g = 9.27 $m/s^2$\\
So the final $t_n^*_{avg}$ = 2.130 s\\
So the final $g_{avg}$ = 8.59 $m/s^2$\\
So the error in g calculated using the equation above is \partial{g} = 0.702 $m/s^2$

\newpage
\subsection{Chinmay Parekh(IMT2020069)}

\begin{table}[h] % [h] forces the table to be output where it is defined in the code (it suppresses floating)
	\centering % Centre the table
	\begin{tabular}{l l l}
		\toprule
		\textbf{H} & \textbf{h} & \textbf{e} \\
		\midrule
		2.00 m & 1.10 m & 0.55\\
        2.00 m & 1.09 m  & 0.54\\
        2.00 m & 1.08 m  & 0.54\\
        2.00 m & 1.10 m & 0.55 \\
        2.00 m & 1.06 m & 0.53 \\
		\bottomrule
	\end{tabular}
	\caption{Observation Table for e}
\end{table}
So the final $h_{avg}$ = 1.09 m\\
So the final $e_{avg}$ = 0.54\\
So the error in e calculated using the equation above is \partial{e} = 0.007

\begin{table}[h]
\centering
\begin{tabular}{||c c c c c c||} 
\toprule
 \hline
 H & t_0 & t_1 & t_2 & t_n^* & g \\ [0.5ex] 
 \midrule
 \hline\hline
 2.00 m & 0.751 s & 0.698 s  & 0.648 s & 2.097 s & 10.51 $m/s^2$  \\ 
 \hline
 2.00 m & 0.787 s & 0.605 s & 0.598 s & 1.989 s & 11.48 $m/s^2$  \\
 \hline
 2.00 m & 0.753 s & 0.690 s & 0.604 s & 2.047 s  & 10.67 $m/s^2$   \\
 \hline
 2.00 m & 0.762 s & 0.685 s & 0.623 s & 2.07 s  & 10.78 $m/s^2$   \\
 \hline
 2.00 m & 0.771 s & 0.690 s & 0.611 s & 2.072 s  & 10.07 $m/s^2$  \\ [1ex]
 \bottomrule
 \hline
\end{tabular}
\caption{Observation Table for g}
\end{table}
Phyphox absolute value of g = 10.03 $m/s^2$\\
So the final $t_n^*_{avg}$ = 2.055 s\\
So the final $g_{avg}$ = 10.70 $m/s^2$\\
So the error in g calculated using the equation above is \partial{g} = 0.935 $m/s^2$\\

\newpage
\subsection{Shashank Shekhar(IMT2020112)}

\begin{table}[h] % [h] forces the table to be output where it is defined in the code (it suppresses floating)
	\centering % Centre the table
	\begin{tabular}{l l l}
		\toprule
		\textbf{H} & \textbf{h} & \textbf{e} \\
		\midrule
		2.00 m & 1.01 m & 0.50\\
        2.00 m & 0.96 m  & 0.48\\
        2.00 m & 0.98 m  & 0.49\\
        2.00 m & 1.05 m & 0.52 \\
        2.00 m & 0.99 m & 0.50 \\
		\bottomrule
	\end{tabular}
	\caption{Observation Table for e}
\end{table}
So the final $h_{avg}$ = 0.99 m\\
So the final $e_{avg}$ = 0.49\\
So the error in e calculated using the equation above is \partial{e} = 0.001

\begin{table}[h]
\centering
\begin{tabular}{||c c c c c c||} 
\toprule
 \hline
 H & t_0 & t_1 & t_2 & t_n^* & g \\ [0.5ex] 
 \midrule
 \hline\hline
 2.00 m & 0.917 s & 0.687 s  & 0.525 s & 2.129 s & 8.57 $m/s^2$  \\ 
 \hline
 2.00 m & 0.889 s & 0.667 s & 0.505 s & 2.061 s & 9.14 $m/s^2$  \\
 \hline 
 2.00 m & 0.914 s & 0.656 s & 0.502 s & 2.072 s  & 9.04 $m/s^2$   \\
 \hline
 2.00 m & 0.912 s & 0.702 s & 0.518 s & 2.132 s  & 8.54 $m/s^2$   \\
 \hline
 2.00 m & 0.917 s & 0.688 s & 0.509 s & 2.114 s  & 8.68 $m/s^2$  \\ 
 [1ex]
 \bottomrule
 \hline
\end{tabular}
\caption{Observation Table for g}
\end{table}
Phyphox absolute value of g = 9.53 $m/s^2$\\
So the final $t_n^*_{avg}$ = 2.102 s\\
So the final $g_{avg}$ = 8.79 $m/s^2$\\
So the error in g calculated using the equation above is \partial{g} = 0.009 $m/s^2$\\

\newpage

\end{document}
